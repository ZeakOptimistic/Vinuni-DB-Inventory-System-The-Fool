\documentclass[a4paper, 12pt]{report}
% \usepackage[utf8]{inputenc}
\usepackage[T1]{fontenc}
% \usepackage[vietnamese]{babel}
\usepackage{lmodern}
\usepackage{amsmath, amssymb, amsfonts}
\usepackage[top=25mm, left=20mm, right=20mm, bottom=25mm]{geometry}
\usepackage{fancyhdr}
\usepackage{titlesec}
\usepackage{booktabs}
\usepackage{indentfirst}
\usepackage{enumitem}
\usepackage{graphicx}
\usepackage{wrapfig}
\usepackage{float}
\usepackage[svgnames]{xcolor}
\usepackage{tikz}
\usepackage{graphicx}
\usepackage{setspace}
\usepackage{float}
\usepackage{array}
\usepackage{tabularx}
\usepackage{lmodern}
\usepackage{hyperref}
\usepackage{listings}
\usepackage{xcolor}


\DeclareUnicodeCharacter{00AD}{} % soft hyphen
\DeclareUnicodeCharacter{FFFD}{} % replacement character


\hypersetup{
    colorlinks=true,
    linkcolor=black,
    urlcolor=blue
}

% \pagestyle{fancy}
% \fancyhf{}
% \fancyfoot[C]{\thepage}
% \fancyfoot[R]{\scriptsize 07/12/2025}
% \renewcommand{\headrulewidth}{0pt}
% \renewcommand{\footrulewidth}{0pt}

\lstdefinestyle{sqlstyle}{
  language=SQL,
  basicstyle=\ttfamily\small,
  keywordstyle=\color{blue}\bfseries,
  stringstyle=\color{teal},
  commentstyle=\color{gray},
  numbers=left,
  numberstyle=\tiny\color{gray},
  stepnumber=1,
  numbersep=8pt,
  showstringspaces=false,
  breaklines=true,
  frame=single,
  tabsize=2,
  captionpos=b
}


\begin{document}

% ================== TITLE PAGE ==================
\begin{titlepage}
    \centering

    \vspace*{2cm}
    {\large VinUniversity\par}

    \vspace{4cm}
    {\huge \textbf{Smart Inventory \& \\ Procurement Management System}\par}

    \vfill
    {\large Tran Quang Khai \par}
    {\large Thai Huu Tri \par}
    {\large Nguyen Ngoc Han \par}
    \vspace{0.5cm}
    {\large COMP3030 - Databases and Database Systems\par}
    \large{Repository: \href{https://github.com/ZeakOptimistic/Vinuni-DB-Inventory-System-The-Fool}{Vinuni DB Inventory System The Fool}}\par
    \vspace{0.4cm}
    {\large Dec 22, 2025\par}

\end{titlepage}

\renewcommand{\thesection}{\arabic{section}}

\tableofcontents
\clearpage

\newpage

% ================== CHAPTER 1: EXECUTIVE SUMMARY ==================
\chapter{Executive Summary}
\section{Project Context and Motivation}
Small retail stores and organizations operating across multiple sites (warehouses and stores) often struggle to maintain accurate, real-time visibility into inventory. Common operational issues include:
(i) stockouts caused by delayed replenishment decisions,
(ii) inconsistent stock records between locations,
(iii) limited insight into which products are selling well over time,
and (iv) manual processes that make auditing and accountability difficult.

To address these challenges, we implemented \textbf{SIPMS (Smart Inventory \& Procurement Management System)}, a MySQL-backed information system with a web interface supporting inventory operations, procurement, sales, inter-location transfers, and reporting.

\section{Objectives and Scope}
The project objectives are:
\begin{itemize}
  \item Provide robust \textbf{entity management} for Products, Categories, Suppliers, Locations, and Users.
  \item Support end-to-end \textbf{procurement workflow} via Purchase Orders (PO) and receiving operations.
  \item Support \textbf{sales workflow} via Sales Orders (SO) with validation and cancellation.
  \item Support \textbf{internal transfers} to move inventory between locations while preventing negative stock.
  \item Provide \textbf{analytics and reporting}, including low-stock monitoring, stock-by-location, top-selling products, and dashboard trend charts.
  \item Enforce \textbf{database-level integrity, security, and auditability} using constraints, triggers, procedures, views, indexes, and MySQL privilege configuration.
\end{itemize}

\section{Deliverables}
The submission includes:
\begin{itemize}
  \item A complete MySQL implementation: tables, views, stored procedures, triggers, indexes, and security grants.
  \item A fully working web application demonstrating CRUD workflows, analytics, reporting, and authentication.
  \item Final report and slide deck PDFs, with screenshots and testing evidence.
  \item A public GitHub repository containing all source code and setup instructions: \href{https://github.com/ZeakOptimistic/Vinuni-DB-Inventory-System-The-Fool}{Vinuni DB Inventory System The Fool}.
\end{itemize}

\section{Document Organization}
This report is organized as follows:
\begin{itemize}
  \item Chapter 2 summarizes the system overview, architecture, and access control.
  \item Chapter 3 demonstrates the implemented web interface modules with screenshots.
  \item Chapter 4 details the database schema and implemented database objects (views, procedures, triggers).
  \item Chapter 5 discusses performance optimizations and security configuration.
  \item Chapter 6 presents testing strategy, sample queries, and validation results.
\end{itemize}

% ========================= CHAPTER 2: SYSTEM OVERVIEW ========================
\chapter{System Overview}

\section{High-Level Architecture}
SIPMS follows a classic three-layer architecture:
\begin{itemize}
  \item \textbf{Frontend (Web UI):} A single-page application that provides module pages for dashboard, entity management, orders, transfers, reports, and user administration.
  \item \textbf{Backend (Application Server):} Provides authenticated APIs and orchestrates business workflows (e.g., creating orders, receiving items, transferring stock, generating reports).
  \item \textbf{Database (MySQL):} Stores normalized transactional data and enforces integrity rules. The database also provides views, triggers, and stored procedures to support reporting and consistent inventory updates.
\end{itemize}

Technology Stack
\begin{itemize}
  \item Database: MySQL 8.x
  \item Backend: Python 3.10+, Django, Django REST Framework, MySQL client, SimpleJWT
  \item Frontend: React 18 + Vite, React Router
  \item Tools: MySQL Workbench, Postman, VS Code
\end{itemize}

\section{User Roles and Access Control}
The system supports role-based access control aligned with operational responsibilities:
\begin{itemize}
  \item \textbf{ADMIN:} full administrative control, including user management and privileged actions. The web UI also supports an administrative \textit{``Login as''} function for testing or support scenarios (see Users module).
  \item \textbf{MANAGER:} operational management access for business entities and order workflows.
  \item \textbf{CLERK:} day-to-day usage for sales, viewing inventory, and basic operations depending on assigned permissions.
\end{itemize}

\begin{table}[htbp]
\centering
\renewcommand{\arraystretch}{1.25}
\begin{tabular}{l c c c}
\toprule
\textbf{Feature} & \textbf{Admin} & \textbf{Manager} & \textbf{Clerk} \\
\cmidrule(lr){1-4}

CRUD Categories / Products      & $\checkmark$ & $\checkmark$ & $\times$ \\
CRUD Suppliers / Locations      & $\checkmark$ & $\checkmark$ & $\times$ \\
Create Purchase Order           & $\checkmark$ & $\checkmark$ & $\times$ \\
Approve / Cancel Purchase Order & $\checkmark$ & $\checkmark$ & $\times$ \\
Receive Goods                   & $\checkmark$ & $\checkmark$ & $\times$ \\
Create Sales Order              & $\checkmark$ & $\checkmark$ & $\checkmark$ \\
Create Stock Transfer           & $\checkmark$ & $\checkmark$ & $\times$ \\
View Dashboard                  & $\checkmark$ & $\checkmark$ & $\checkmark$ \\
Manage Users / Roles            & $\checkmark$ & $\times$     & $\times$ \\

\bottomrule
\end{tabular}
\caption{Role-Based Access Control Matrix}
\label{tab:rbac-matrix}
\end{table}

\section{Core Data Concepts}
SIPMS is built around these core concepts:
\begin{itemize}
  \item Products are uniquely identified by SKU, optionally barcode, and belong to exactly one category.
  \item Inventory is tracked per (product, location) pair with an explicit quantity on hand.
  \item Purchase orders increase stock upon receiving; sales orders decrease stock upon confirmation; transfers move stock between locations.
  \item Sales orders reduce inventory when confirmed; in the web app this confirmation is performed automatically during the sales order workflow (rather than a separate manual step).
  \item In report, low stock is determined by comparing quantity on hand vs reorder level; top-selling products are computed from confirmed sales orders in the last 30 days.
\end{itemize}

% =========================CHAPTER 3: Web Interface and Features Demonstration ================================
\chapter{Web Interface and Features Demonstration}

\section{UI Layout and Navigation}
The SIPMS web interface uses a consistent layout across modules:
\begin{itemize}
  \item A left sidebar navigation with pages: Dashboard, Products, Suppliers, Locations, Categories, Purchase Orders, Sales Orders, Transfers, Reports, and Users. Visible pages and available actions vary by role (Admin/Manager/Clerk).
  \item A top header displaying the signed-in user and role, along with a Logout action.
  \item Each module page provides search/filter controls, data tables, and context-specific actions.
\end{itemize}

\section{Dashboard Page}
The Dashboard provides operational overview and analytics widgets designed for daily monitoring:
\begin{itemize}
  \item \textbf{Role banner:} displays current signed-in role (e.g., Admin).
  \item \textbf{KPI cards:} summary counts for Products (active vs inactive), Purchase Orders (open vs closed), Sales Orders (confirmed vs cancelled), and an \textbf{estimated stock value}.
  \item \textbf{Time range selector:} a dropdown (e.g., ``Last 30 days'') controls time-based analytics on the dashboard.
  \item \textbf{Orders trend chart:} visualizes purchase and sales order activity over time.
  \item \textbf{Sales revenue trend chart:} visualizes revenue fluctuations and highlights total revenue and peak values over the selected range.
  \item \textbf{Highlights:}
    (i) low stock highlights (with severity labels such as LOW and ``Need +X''), and
    (ii) top selling highlights for quick attention, each linking to the Reports module.
\end{itemize}

\begin{figure}[htbp]
    \centering
    \includegraphics[width=1\textwidth]{demo/dashboard1.png}
    \caption{Dashboard Page Screenshot}
\end{figure}

\begin{figure}[htbp]
    \centering
    \includegraphics[width=1\textwidth]{demo/dashboard2.png}
    \caption{Interact with Chart}
\end{figure}

\begin{figure}[htbp]
    \centering
    \includegraphics[width=1\textwidth]{demo/dashboard3.png}
    \caption{Interact with Chart and Option Selector for Time Range}
\end{figure}

\begin{figure}[htbp]
    \centering
    \includegraphics[width=1\textwidth]{demo/dashboard4.png}
    \caption{The Modification of Time Range Affects the Charts}
\end{figure}

\begin{figure}[htbp]
    \centering
    \includegraphics[width=1\textwidth]{demo/dashboard5.png}
    \caption{Low Stock and Top Selling Highlights}
\end{figure}

\newpage

\section{Products Page}
The Products page supports full lifecycle management of the product catalog:
\begin{itemize}
  \item Create new products via a form capturing: SKU, Name, Category, Unit Price, Unit, Reorder Level, and Status (Active/Inactive).
  \item Search by product name, SKU, or barcode.
  \item Sorting (e.g., by name A$\rightarrow$Z).
  \item Tabular view including: SKU, Name, Category, Unit Price, Unit, Reorder Level, Status, Created At, and Actions.
  \item Actions include editing and status toggling (Activate/Deactivate).
  \item Pagination controls including a \textbf{page size selector} (e.g., 10 per page), and Previous/Next navigation.
\end{itemize}

\begin{figure}[htbp]
    \centering
    \includegraphics[width=1\textwidth]{demo/product1.png}
    \caption{Product Page Screenshot}
\end{figure}

\begin{figure}[htbp]
    \centering
    \includegraphics[width=1\textwidth]{demo/product2.png}
    \caption{Edit Existing Product Form}
\end{figure}

\begin{figure}[htbp]
    \centering
    \includegraphics[width=1\textwidth]{demo/product3.png}
    \caption{Sorting Feature}
\end{figure}

\begin{figure}[htbp]
    \centering
    \includegraphics[width=1\textwidth]{demo/product4.png}
    \caption{Searching Feature}
\end{figure}


\begin{figure}[htbp]
    \centering
    \includegraphics[width=1\textwidth]{demo/product5.png}
    \caption{Pagination Feature}
\end{figure}


\begin{figure}[htbp]
    \centering
    \includegraphics[width=1\textwidth]{demo/product6.png}
    \caption{Notification Deactivate Product}
\end{figure}


\begin{figure}[htbp]
    \centering
    \includegraphics[width=1\textwidth]{demo/product7.png}
    \caption{Notification Activate Product}
\end{figure}

\begin{figure}[htbp]
    \centering
    \includegraphics[width=1\textwidth]{demo/product8.png}
    \caption{Create New Product Form}
\end{figure}

\newpage
\section{Suppliers Page}
The Suppliers module manages vendor information required for procurement:
\begin{itemize}
  \item Create new suppliers via a form capturing: Name, Contact Person, Phone, Email, Address, Payment Terms, and Status (Active/Inactive).
  \item Search by supplier name, phone, or email.
  \item Sorting (e.g., by name A$\rightarrow$Z).
  \item Supplier table includes: Name, Contact, Phone, Email, Address, Payment Terms, Status, and Actions.
  \item Actions include Edit, Activate/Deactivate, and Delete.
  \item Pagination with page size selector.
\end{itemize}

\begin{figure}[htbp]
    \centering
    \includegraphics[width=1\textwidth]{demo/supplier1.png}
    \caption{Suppliers Page Screenshot}
\end{figure}

\begin{figure}[htbp]
    \centering
    \includegraphics[width=1\textwidth]{demo/supplier2.png}
    \caption{Edit Existing Supplier Form}
\end{figure}

\begin{figure}[htbp]
    \centering
    \includegraphics[width=1\textwidth]{demo/supplier3.png}
    \caption{Sorting Feature}
\end{figure}

\begin{figure}[htbp]
    \centering
    \includegraphics[width=1\textwidth]{demo/supplier4.png}
    \caption{Searching Feature}
\end{figure}

\begin{figure}[htbp]
    \centering
    \includegraphics[width=1\textwidth]{demo/supplier5.png}
    \caption{Notification Deactivate Supplier}
\end{figure}

\begin{figure}[htbp]
    \centering
    \includegraphics[width=1\textwidth]{demo/supplier6.png}
    \caption{Notification Activate Supplier}
\end{figure}

\begin{figure}[htbp]
    \centering
    \includegraphics[width=1\textwidth]{demo/supplier7.png}
    \caption{Pagination Feature}
\end{figure}

\begin{figure}[htbp]
    \centering
    \includegraphics[width=1\textwidth]{demo/supplier8.png}
    \caption{Create New Supplier Form}
\end{figure}

\newpage

\section{Locations Page}
Locations represent warehouses and stores and are used for inventory tracking and transfers:
\begin{itemize}
  \item Create new locations via a form capturing: Name, Type (STORE / WAREHOUSE), Address, and Status (Active/Inactive).
  \item Search by location name or address.
  \item Sorting by name.
  \item Location table includes: Name, Type (STORE / WAREHOUSE), Address, Status, Created At, and Actions.
  \item Actions include Edit, Activate/Deactivate, and Delete.
  \item Pagination with page size selector.
\end{itemize}


\begin{figure}[htbp]
    \centering
    \includegraphics[width=1\textwidth]{demo/location1.png}
    \caption{Locations Page Screenshot}
\end{figure}

\begin{figure}[htbp]
    \centering
    \includegraphics[width=1\textwidth]{demo/location2.png}
    \caption{Edit Existing Location Form}
\end{figure}

\begin{figure}[htbp]
    \centering
    \includegraphics[width=1\textwidth]{demo/location3.png}
    \caption{Sorting Feature}
\end{figure}

\begin{figure}[htbp]
    \centering
    \includegraphics[width=1\textwidth]{demo/location4.png}
    \caption{Searching Feature}
\end{figure}

\begin{figure}[htbp]
    \centering
    \includegraphics[width=1\textwidth]{demo/location5.png}
    \caption{Notification Deactivate Location}
\end{figure}

\begin{figure}[htbp]
    \centering
    \includegraphics[width=1\textwidth]{demo/location6.png}
    \caption{Notification Activate Location}
\end{figure}

\begin{figure}[htbp]
    \centering
    \includegraphics[width=1\textwidth]{demo/location7.png}
    \caption{Pagination Feature}
\end{figure}

\begin{figure}[htbp]
    \centering
    \includegraphics[width=1\textwidth]{demo/location8.png}
    \caption{Create New Supplier Form}
\end{figure}

\newpage

\section{Categories Page}
Categories provide product classification and support filtering and reporting consistency:
\begin{itemize}
  \item Create new categories via a form capturing: Name, Description, and Status (Active/Inactive).
  \item Sorting by name.
  \item Search by category name.
  \item Category table includes: Name, Description, Status, and Actions.
  \item Actions include Edit, Activate/Deactivate, and Delete.
  \item Pagination with page size selector.
\end{itemize}


\begin{figure}[htbp]
    \centering
    \includegraphics[width=1\textwidth]{demo/category1.png}
    \caption{Categories Page Screenshot}
\end{figure}

\begin{figure}[htbp]
    \centering
    \includegraphics[width=1\textwidth]{demo/category2.png}
    \caption{Edit Existing Category Form}
\end{figure}

\begin{figure}[htbp]
    \centering
    \includegraphics[width=1\textwidth]{demo/category3.png}
    \caption{Sorting Feature}
\end{figure}

\begin{figure}[htbp]
    \centering
    \includegraphics[width=1\textwidth]{demo/category4.png}
    \caption{Searching Screenshot}
\end{figure}

\begin{figure}[htbp]
    \centering
    \includegraphics[width=1\textwidth]{demo/category5.png}
    \caption{Notification Deactivate Category}
\end{figure}

\begin{figure}[htbp]
    \centering
    \includegraphics[width=1\textwidth]{demo/category6.png}
    \caption{Notification Activate Category}
\end{figure}

\begin{figure}[htbp]
    \centering
    \includegraphics[width=1\textwidth]{demo/category7.png}
    \caption{Pagination Feature}
\end{figure}

\begin{figure}[htbp]
    \centering
    \includegraphics[width=1\textwidth]{demo/category8.png}
    \caption{Create New Supplier Form}
\end{figure}

\newpage

\section{Purchase Orders Page}
The Purchase Orders page supports procurement tracking and receiving operations:
\begin{itemize}
  \item Filters include supplier filter, location filter, and status filter.
  \item Purchase order table includes: PO \#, Supplier, Location, Status, Order Date, Total, Created At, Items, and Actions.
  \item ``View items (N)'' opens detailed line items for the selected PO.
  \item ``Receive all'' is available when the order is in a receivable state (e.g., Approved) and disabled when not applicable (e.g., Closed).
  \item Create/Receive actions are available to Admin/Manager; Clerk is view-only
  \item Pagination with page size selector.
\end{itemize}

\begin{figure}[htbp]
    \centering
    \includegraphics[width=1\textwidth]{demo/po1.png}
    \caption{Purchase Orders Page Screenshot}
\end{figure}

\begin{figure}[htbp]
    \centering
    \includegraphics[width=1\textwidth]{demo/po2.png}
    \caption{Create New Purchase Order Form}
\end{figure}

\begin{figure}[htbp]
    \centering
    \includegraphics[width=1\textwidth]{demo/po3.png}
    \caption{Show Items of a Purchase Order}
\end{figure}

\begin{figure}[htbp]
    \centering
    \includegraphics[width=1\textwidth]{demo/po4.png}
    \caption{Receive All Items from a Purchase Order}
\end{figure}

\begin{figure}[htbp]
    \centering
    \includegraphics[width=1\textwidth]{demo/po5.png}
    \caption{Filtering Status Feature}
\end{figure}

\begin{figure}[htbp]
    \centering
    \includegraphics[width=1\textwidth]{demo/po6.png}
    \caption{Pagination Feature}
\end{figure}

\begin{figure}[htbp]
    \centering
    \includegraphics[width=1\textwidth]{demo/po7.png}
    \includegraphics[width=1\textwidth]{demo/po8.png}
    \caption{Searching Feature}
\end{figure}

\newpage

\section{Sales Orders Page}
The Sales Orders page supports sales tracking and order cancellation:
\begin{itemize}
  \item Filters include customer filter, location filter, and status filter.
  \item Sales order table includes: SO \#, Customer, Location, Status, Order Date, Total, Created At, Items, and Actions.
  \item ``View items (N)'' opens detailed line items.
  \item ``Cancel'' action is provided for orders in a cancellable state; cancelled orders display status accordingly.
  \item On confirmation/creation, the system validates stock sufficiency to prevent negative inventory; insufficient stock returns an error.
  \item Pagination with page size selector.
\end{itemize}


\begin{figure}[htbp]
    \centering
    \includegraphics[width=1\textwidth]{demo/so1.png}
    \caption{Sales Orders Page Screenshot}
\end{figure}

\begin{figure}[htbp]
    \centering
    \includegraphics[width=1\textwidth]{demo/so2.png}
    \caption{Create New Sales Order Form}
\end{figure}

\begin{figure}[htbp]
    \centering
    \includegraphics[width=1\textwidth]{demo/so3.png}
    \caption{Show Items of a Sales Order}
\end{figure}

\begin{figure}[htbp]
    \centering
    \includegraphics[width=1\textwidth]{demo/so4.png}
    \caption{Notification Cancel a Sales Order}
\end{figure}

\begin{figure}[htbp]
    \centering
    \includegraphics[width=1\textwidth]{demo/so5.png}
    \caption{Status After Cancelling a Sales Order}
\end{figure}

\begin{figure}[htbp]
    \centering
    \includegraphics[width=1\textwidth]{demo/so6.png}
    \caption{Pagination Feature}
\end{figure}

\begin{figure}[htbp]
    \centering
    \includegraphics[width=1\textwidth]{demo/so7.png}
    \caption{Filtering Status Feature}
\end{figure}

\begin{figure}[htbp]
    \centering
    \includegraphics[width=1\textwidth]{demo/so8.png}
    \caption{Searching Feature}
\end{figure}

\newpage

\section{Transfers Page}
The Transfers module supports internal movement of stock across locations:
\begin{itemize}
  \item A transfer form includes: Product selection, Quantity input, From Location, and To Location.
  \item A ``Transfer stock'' action executes the transfer.
  \item A recent transfers table provides operational traceability, including:
        Transfer \#, Product, From, To, Quantity, From qty after, and To qty after.
  \item Pagination with page size selector for transfer history.
\end{itemize}

\begin{figure}[htbp]
    \centering
    \includegraphics[width=1\textwidth]{demo/transfer1.png}
    \caption{Transfers Page Screenshot}
\end{figure}

\begin{figure}[htbp]
    \centering
    \includegraphics[width=1\textwidth]{demo/transfer2.png}
    \caption{Filtering Product Feature}
\end{figure}

\begin{figure}[htbp]
    \centering
    \includegraphics[width=1\textwidth]{demo/transfer3.png}
    \caption{Filtering From Location Feature}
\end{figure}

\begin{figure}[htbp]
    \centering
    \includegraphics[width=1\textwidth]{demo/transfer4.png}
    \caption{Filtering To Location Feature}
\end{figure}

\begin{figure}[htbp]
    \centering
    \includegraphics[width=1\textwidth]{demo/transfer5.png}
    \caption{Notification Successful Stock Transfer}
\end{figure}

\begin{figure}[htbp]
    \centering
    \includegraphics[width=1\textwidth]{demo/transfer6.png}
    \caption{Notification Failed Stock Transfer due to Insufficient Stock}
\end{figure}

\begin{figure}[htbp]
    \centering
    \includegraphics[width=1\textwidth]{demo/transfer7.png}
    \caption{Pagination Feature}
\end{figure}


\newpage

\section{Reports Page}
The Reports module provides three main reporting modes using tab navigation:
\begin{itemize}
  \item \textbf{Low stock:} lists product-location pairs where quantity on hand is below reorder level.
  \item \textbf{Stock per location:} shows inventory distribution across locations.
  \item \textbf{Top selling (30 days):} ranks products by quantity sold and/or revenue (based on confirmed sales orders in the last 30 days).
\end{itemize}

\noindent Common controls include:
\begin{itemize}
  \item Search by product name or SKU.
  \item Location filter (e.g., ``All locations'').
  \item Paginated tables with page size selector.
\end{itemize}

\begin{figure}[htbp]
    \centering
    \includegraphics[width=1\textwidth]{demo/report1.png}
    \caption{Low Stock Tab}
\end{figure}

\begin{figure}[htbp]
    \centering
    \includegraphics[width=1\textwidth]{demo/report2.png}
    \caption{Pagination Feature}
\end{figure}

\begin{figure}[htbp]
    \centering
    \includegraphics[width=1\textwidth]{demo/report3.png}
    \caption{Filtering Locations Feature}
\end{figure}

\begin{figure}[htbp]
    \centering
    \includegraphics[width=1\textwidth]{demo/report4.png}
    \caption{Searching Feature}
\end{figure}

\begin{figure}[htbp]
    \centering
    \includegraphics[width=1\textwidth]{demo/report5.png}
    \caption{Top Selling Tab}
\end{figure}

\begin{figure}[htbp]
    \centering
    \includegraphics[width=1\textwidth]{demo/report6.png}
    \caption{Stock per Location Tab}
\end{figure}

\begin{figure}[htbp]
    \centering
    \includegraphics[width=1\textwidth]{demo/report7.png}
    \caption{Searching and Filtering Locations Feature}
\end{figure}

\begin{figure}[htbp]
    \centering
    \includegraphics[width=1\textwidth]{demo/report8.png}
    \caption{Searching Feature}
\end{figure}

\newpage

\section{Users Page}
The Users module provides administrative management of application accounts:
\begin{itemize}
  \item Search by username, name, or email; and sorting (e.g., ``Newest'').
  \item Table fields include: Username, Full name, Email, Role, Status, and Actions.
  \item Actions include: Edit, \textbf{Login as}, Activate/Deactivate, and Delete.
  \item The ``Login as'' capability supports testing and support by allowing an admin to switch context to another user account.
\end{itemize}

\begin{figure}[htbp]
    \centering
    \includegraphics[width=1\textwidth]{demo/user1.png}
    \caption{Users Page Screenshot}
\end{figure}

\begin{figure}[htbp]
    \centering
    \includegraphics[width=1\textwidth]{demo/user2.png}
    \caption{Edit Existing User Form}
\end{figure}

\begin{figure}[htbp]
    \centering
    \includegraphics[width=1\textwidth]{demo/user3.png}
    \caption{Notification Deactivate User}
\end{figure}

\begin{figure}[htbp]
    \centering
    \includegraphics[width=1\textwidth]{demo/user4.png}
    \caption{Login As Feature Without Password if Admin Role}
\end{figure}

\begin{figure}[htbp]
    \centering
    \includegraphics[width=1\textwidth]{demo/user5.png}
    \caption{Notification Failed to Login As if User is Inactive}
\end{figure}

\begin{figure}[htbp]
    \centering
    \includegraphics[width=1\textwidth]{demo/user6.png}
    \caption{Notification Activate User}
\end{figure}

\begin{figure}[htbp]
    \centering
    \includegraphics[width=1\textwidth]{demo/user7.png}
    \caption{Pagination Feature}
\end{figure}

\begin{figure}[htbp]
    \centering
    \includegraphics[width=1\textwidth]{demo/user8.png}
    \caption{Sorting Feature}
\end{figure}

\begin{figure}[htbp]
    \centering
    \includegraphics[width=1\textwidth]{demo/user9.png}
    \caption{Searching Feature}
\end{figure}

\begin{figure}[htbp]
    \centering
    \includegraphics[width=1\textwidth]{demo/user10.png}
    \caption{Create New User Form}
\end{figure}

\begin{figure}[htbp]
    \centering
    \includegraphics[width=1\textwidth]{demo/user11.png}
    \caption{Login As Feature with Active User}
\end{figure}

\begin{figure}[htbp]
    \centering
    \includegraphics[width=1\textwidth]{demo/user12.png}
    \caption{Screenshot After Using Login As Feature with Admin Role and Full Access}
\end{figure}

\begin{figure}[htbp]
    \centering
    \includegraphics[width=1\textwidth]{demo/user13.png}
    \caption{Screenshot After Using Login As Feature with Manager Role and Limited Access}
\end{figure}

\begin{figure}[htbp]
    \centering
    \includegraphics[width=1\textwidth]{demo/user14.png}
    \caption{Screenshot After Using Login As Feature with Clerk Role and Very Limited Access}
\end{figure}

\begin{figure}[htbp]
    \centering
    \includegraphics[width=1\textwidth]{demo/user15.png}
    \caption{Notification Delete User}
\end{figure}

\begin{figure}[htbp]
    \centering
    \includegraphics[width=1\textwidth]{demo/user16.png}
    \caption{Notification Failed to Delete User if User is referring to Existing Records}
\end{figure}
\newpage
\section{Authentication}
SIPMS enforces authentication before a user can access any business module (Dashboard,
Products, Purchase Orders, Sales Orders, Reports, and User Management). This design
prevents unauthorized access to inventory and purchasing data, and it ensures that every
action is associated with a specific user account and role.

\begin{figure}[htbp]
    \centering
    \includegraphics[width=1\textwidth]{demo/login.png}
    \caption{Login Page Screenshot}
\end{figure}

\newpage
%==============CHAPTER 4: Database Implementation============

\chapter{Database Implementation}
\label{chap:db-implementation}

This chapter documents the database layer of SIPMS, including the implemented schema,
constraints, views, stored procedures, triggers, indexes, and security configuration.
All database artifacts are provided as SQL scripts in the project repository. In this report,
we focus on what was implemented and why, and we include the key SQL scripts as
listings for completeness.

% --------------------------------------------------------------------
\section{Implementation Artifacts and Execution Order}
\label{sec:db-artifacts-order}

\subsection{SQL scripts}
The database implementation is organized into the following scripts:

\begin{itemize}
  \item \texttt{schema.sql}: core schema (database creation, tables, keys, constraints) and a small set of baseline indexes.
  \item \texttt{seed\_data.sql}: demo dataset (users/products/locations/orders) used for testing.
  \item \texttt{views.sql}: reporting/analytics views used by dashboard and reports.
  \item \texttt{procedures.sql}: stored procedures for critical workflows (purchase order creation and sales confirmation).
  \item \texttt{triggers.sql}: triggers enforcing inventory consistency and preventing invalid stock movements.
  \item \texttt{indexes.sql}: additional indexes for performance and common query patterns.
  \item \texttt{security.sql}: database users and privilege configuration (least privilege).
\end{itemize}

\subsection{Recommended execution order}
A typical setup sequence is:

\begin{enumerate}
  \item Run \texttt{schema.sql} to create the database (\texttt{sipms}), tables, and constraints.
  \item Run \texttt{seed\_data.sql} to load demo/testing records.
  \item Run \texttt{views.sql} to create reporting views.
  \item Run \texttt{procedures.sql} to create stored procedures.
  \item Run \texttt{triggers.sql} to activate automatic inventory enforcement.
  \item Run \texttt{indexes.sql} to add performance indexes.
  \item Run \texttt{security.sql} to create database users and grant privileges.
\end{enumerate}

\paragraph{Database engine and encoding.}
The schema is designed for MySQL and initializes the database with \texttt{utf8mb4} to
support Unicode content (e.g., product names, supplier information) consistently.

% --------------------------------------------------------------------
\section{Logical Data Model Summary}
\label{sec:db-logical-model}

\begin{figure}[H]
  \centering
  \includegraphics[width=\textwidth]{diagram sipms.png}
  \label{fig:sipms-er}
\end{figure}

\newpage

Figure~\ref{fig:sipms-er} illustrates the main entities and relationships used in SIPMS.

\subsection{ER diagram and relationships}
The SIPMS database follows a relational design separating:
(i) master data (products, suppliers, locations),
(ii) transactional documents (purchase orders, sales orders),
and (iii) inventory state and auditability (inventory level, stock movement, audit log).

\subsection{Core tables by domain}
\paragraph{Identity and roles.}
\begin{itemize}
  \item \texttt{role}: role definitions (e.g., ADMIN, MANAGER, CLERK).
  \item \texttt{app\_user}: application users, linked to \texttt{role} via foreign key.
\end{itemize}

\paragraph{Master data.}
\begin{itemize}
  \item \texttt{category}: product categories.
  \item \texttt{product}: product master (SKU, barcode, pricing, reorder level), linked to \texttt{category}.
  \item \texttt{supplier}: supplier master data used in procurement.
  \item \texttt{location}: physical locations (warehouse/store) where inventory is tracked.
\end{itemize}

\paragraph{Procurement and sales documents.}
\begin{itemize}
  \item \texttt{purchase\_order} and \texttt{purchase\_order\_item}: purchase orders and their line items.
  \item \texttt{sales\_order} and \texttt{sales\_order\_item}: sales orders and their line items.
\end{itemize}

\paragraph{Inventory and audit trail.}
\begin{itemize}
  \item \texttt{stock\_movement}: immutable event log of stock changes (receipts, sales issues, transfers, adjustments).
  \item \texttt{inventory\_level}: current on-hand quantity per \texttt{(product, location)} pair.
  \item \texttt{audit\_log}: optional audit entries that can record user actions and descriptions.
\end{itemize}

% --------------------------------------------------------------------
\section{Constraints and Integrity Rules}
\label{sec:db-integrity}

SIPMS enforces data integrity at the database layer to avoid inconsistent inventory states
and to preserve referential correctness between entities.

\subsection{Keys and relationships}
\begin{itemize}
  \item Primary keys are used for all master and document header tables (e.g., \texttt{product\_id}, \texttt{po\_id}, \texttt{so\_id}).
  \item Composite primary keys are used for weak entities:
        \texttt{inventory\_level(product\_id, location\_id)},
        \texttt{purchase\_order\_item(po\_id, product\_id)},
        \texttt{sales\_order\_item(so\_id, product\_id)}.
  \item Foreign keys enforce valid references (e.g., order headers reference existing suppliers/locations/users; order items reference existing products).
\end{itemize}

\subsection{Domain constraints and statuses}
Several attributes use constrained domains to ensure predictable application behavior:
\begin{itemize}
  \item Status fields (e.g., purchase order status, sales order status) are restricted to predefined values.
  \item Quantities are constrained to valid ranges (e.g., order quantities must be positive; received quantity must be non-negative).
\end{itemize}

\subsection{Non-negative inventory invariant}
A critical business rule is that inventory on-hand must not become negative. SIPMS enforces
this invariant directly in the database using triggers on \texttt{stock\_movement} (described in
Section~\ref{sec:db-triggers}), so that any process inserting movements (API or other tools)
cannot reduce inventory below zero.

% --------------------------------------------------------------------
\section{Inventory Consistency Design}
\label{sec:db-inventory-design}

\subsection{Stock movement as the system of record}
SIPMS records each inventory change as a row in \texttt{stock\_movement}. This provides:
\begin{itemize}
  \item a complete audit trail of what changed, when, where, and by whom,
  \item a unified model for multiple operations (purchase receipt, sales issue, transfer, adjustment),
  \item a reliable base for reporting and troubleshooting.
\end{itemize}

\subsection{Current inventory snapshot}
The table \texttt{inventory\_level} stores the current on-hand quantity for each
\texttt{(product, location)} pair. Instead of updating inventory levels from many places,
SIPMS updates \texttt{inventory\_level} automatically whenever a new stock movement is inserted.
This reduces duplicated logic and helps keep the snapshot consistent with the movement ledger.

% --------------------------------------------------------------------
\section{Database Views for Reporting}
\label{sec:db-views}

To simplify reporting queries and reduce repeated joins in the application layer, SIPMS
implements views focused on common dashboard/reporting needs.

\subsection{Implemented views}
\begin{itemize}
  \item \texttt{view\_stock\_per\_location}: inventory per product per location, including reorder-level flags and stock value.
  \item \texttt{view\_low\_stock\_products}: subset of products below reorder level (supports ``Low Stock'' dashboard/report).
  \item \texttt{view\_top\_selling\_products\_last\_30\_days}: aggregated sales quantities and revenue for confirmed sales in the last 30 days.
\end{itemize}

\subsection{Benefits}
Using views improves maintainability and consistency:
\begin{itemize}
  \item Reporting logic (joins and filters) is centralized in the database.
  \item The frontend/backend can query a stable interface for analytics needs.
  \item Performance is improved when combined with appropriate indexes (Section~\ref{subsec:indexing-strategy}).
\end{itemize}

\subsection{SQL Listing: \texttt{views.sql}}
\lstinputlisting[style=sqlstyle, language=SQL,label={lst:views}]{../../database/views.sql}

% --------------------------------------------------------------------
\section{Stored Procedures for Critical Workflows}
\label{sec:db-procedures}

Stored procedures are used to centralize multi-step operations that must follow strict business
rules and validation.

\subsection{\texttt{sp\_create\_purchase\_order}}
This procedure creates a new purchase order header with an initial status and total amount.
Centralizing purchase order creation reduces duplicated logic across the application and provides
a consistent initial state for newly created purchase orders.
The procedure also returns the newly created purchase order identifier to the caller, enabling the
application to attach line items or continue the workflow.

\subsection{\texttt{sp\_confirm\_sales\_order}}
Confirming a sales order must ensure inventory sufficiency before reducing stock.
This procedure enforces the workflow:
\begin{itemize}
  \item Load and lock the sales order context (location information).
  \item Check inventory availability for each sales line item at the sales location.
  \item If any item is insufficient, reject the confirmation by raising a database error.
  \item Otherwise, update the sales order status to \texttt{CONFIRMED}.
  \item Insert \texttt{stock\_movement} rows (e.g., \texttt{SALES\_ISSUE}) for each order line item so that inventory is reduced consistently.
\end{itemize}

\paragraph{Error handling.}
perfoThe procedure signals clear validation failures (e.g., missing sales order or insufficient stock).
These errors are propagated to the API layer and displayed as user-facing validation messages.

\subsection{SQL Listing: \texttt{procedures.sql}}
\lstinputlisting[style=sqlstyle, language=SQL,label={lst:procedures}]{../../database/procedures.sql}

% --------------------------------------------------------------------
\section{Triggers Enforcing Inventory Rules}
\label{sec:db-triggers}

Database triggers are used to enforce inventory correctness regardless of which component inserts
stock movement rows.

\subsection{Validation trigger: quantity must be positive}
A \texttt{BEFORE INSERT} trigger ensures that a stock movement quantity is strictly positive.
This prevents invalid movements from entering the system and simplifies downstream logic
(because movement direction is determined by movement type, not by negative quantities).

\subsection{Consistency trigger: update inventory level and prevent negative stock}
An \texttt{AFTER INSERT} trigger updates \texttt{inventory\_level} based on the inserted movement:
\begin{itemize}
  \item The movement type determines whether the inventory delta is positive (e.g., purchase receipt)
        or negative (e.g., sales issue).
  \item The trigger locks the current inventory row for the affected \texttt{(product, location)} pair to
        avoid race conditions under concurrent updates.
  \item If the new computed quantity would be negative, the trigger rejects the insert operation by
        raising an error.
  \item Otherwise, the trigger inserts or updates the corresponding \texttt{inventory\_level} row.
\end{itemize}

This design ensures that the inventory snapshot cannot silently diverge from the movement ledger,
and that negative inventory is prevented at the database layer.

\subsection{SQL Listing: \texttt{triggers.sql}}
\lstinputlisting[style=sqlstyle, language=SQL,label={lst:triggers}]{../../database/triggers.sql}

% ---------------------------
\section{Deployment Notes and Verification Checklist}
\label{sec:db-deploy-verify}

\subsection{Setup checklist}
After executing the scripts in the recommended order, the following checks confirm correct deployment:
\begin{itemize}
  \item All core tables exist and foreign keys are created successfully.
  \item Views are queryable and return expected columns.
  \item Stored procedures are callable and return expected results/errors.
  \item Stock movement insertion updates \texttt{inventory\_level} automatically.
  \item Negative inventory and invalid quantities are rejected by triggers.
\end{itemize}

\subsection{Integration with the application}
The backend uses the database as the source of truth for inventory and leverages:
\begin{itemize}
  \item views for fast reporting queries,
  \item stored procedures for critical workflow validation,
  \item triggers for automatic and consistent inventory snapshot updates.
\end{itemize}

% --------------------------------------------------------------------
\section{Limitations and Future Improvements}
\label{sec:db-limitations}

While the current database implementation satisfies the core project requirements, the following
improvements are natural next steps:
\begin{itemize}
  \item Strengthen transactional atomicity for complex workflows by ensuring multi-step status updates
        and stock movement inserts commit as a single unit in all execution paths.
  \item Extend auditing by writing \texttt{audit\_log} entries for key operations (e.g., confirm sales, receive goods,
        transfers) either in procedures/triggers or consistently at the API service layer.
  \item Add additional analytical views (e.g., inventory valuation over time, supplier performance, gross margin by product).
\end{itemize}

\newpage

% ================CHAPTER 5==========================================
\chapter{Performance Optimization and Security Configuration}
\label{chap:perf-security}

This chapter describes the main performance considerations and the security model of SIPMS.
While Chapter~4 focuses on database objects (schema, views, procedures, triggers), here we
explain why specific indexes and privilege configurations were introduced, and how they
support the web application's workload.

\section{Performance Optimization}
\label{sec:performance}

\subsection{Workload characteristics and query patterns}
SIPMS is an operational system accessed through a web interface. The most frequent query
patterns come from typical UI interactions:

\begin{itemize}
  \item \textbf{Search and filtering:} products searched by SKU/name; orders filtered by date/status;
  reports filtered by location and keyword.
  \item \textbf{Pagination and recent-first listing:} module tables frequently load ``latest $N$'' records
  (e.g., newest orders, most recent transfers/movements).
  \item \textbf{Analytics aggregates:} dashboard and reports require grouped aggregations such as
  low-stock lists and top-selling products over a time window (e.g., last 30 days).
  \item \textbf{Join-heavy reads:} UI pages often join transactional tables with master data (product,
  location, supplier) to render readable names and derived values.
\end{itemize}

Because these patterns are repeated across modules, we focus on reducing full table scans,
improving join performance, and supporting fast time-window filtering for analytics.

\subsection{Indexing strategy}
\label{subsec:indexing-strategy}

Indexes are selected based on the most common access paths:

\begin{itemize}
  \item \textbf{Product lookup and search.}
  SKU-based search is a primary workflow in inventory systems, so an index on
  \texttt{product(sku)} accelerates filtering and exact lookup. If you also support frequent
  name search/sort at scale, an additional index on \texttt{product(name)} can be considered.

  \item \textbf{Inventory snapshots by location.}
  Inventory is stored per \texttt{(product\_id, location\_id)} and typically uses a composite
  primary key on these columns, which already supports many join patterns. If the workload
  frequently filters by location first, an additional index (e.g., \texttt{(location\_id, product\_id)})
  can further improve location-centric queries.

  \item \textbf{Movement history and audit trails.}
  Operational traceability requires reading movements by time range and by product/location.
  A composite index on \texttt{stock\_movement(movement\_date, product\_id)} improves recent-history
  queries and time-window reporting.

  \item \textbf{Dashboard time-series queries.}
  Dashboard charts commonly filter orders by date and status. Composite indexes on
  \texttt{sales\_order(order\_date, status)} and \texttt{purchase\_order(order\_date, status)} improve
  these time-based scans and avoid unnecessary table reads.
\end{itemize}

\subsection{SQL Listing: \texttt{indexes.sql}}
\lstinputlisting[style=sqlstyle, language=SQL,label={lst:indexes}]{../../database/indexes.sql}

\subsection{Query design optimizations at the application layer}
Besides database indexes, SIPMS applies several practical query-level optimizations:

\begin{itemize}
  \item \textbf{Pagination / limits:} API endpoints cap returned records (e.g., ``last $N$'' orders or transfers)
  to reduce response size and keep UI interactions responsive.
  \item \textbf{Efficient joins:} frequently used foreign-key relations are retrieved with ORM join helpers
  (e.g., \texttt{select\_related}) to reduce the number of round trips.
  \item \textbf{Pre-validation to reduce failures:} for operations like transfers and confirmations, the
  application performs basic checks early (e.g., locations must differ, current stock sufficiency),
  while still relying on database enforcement for correctness.
\end{itemize}

\subsection{Use of views for reporting performance}
Reporting pages are powered by database views that predefine joins and derived attributes
(e.g., stock value, below-reorder flag). This design has two benefits:

\begin{itemize}
  \item \textbf{Consistency:} report calculations are standardized in SQL rather than duplicated across
  endpoints.
  \item \textbf{Performance:} the database optimizer can reuse indexes and execute a well-defined plan
  for repeated reporting queries.
\end{itemize}

\subsection{Performance verification (recommended evidence)}
To strengthen this chapter, include 1--2 examples of evidence (you can add screenshots):

\begin{itemize}
  \item \textbf{EXPLAIN plans} for key queries (product lookup, low-stock report, top-selling aggregate).
  \item \textbf{Observed response time} or before/after comparison when indexes are enabled.
\end{itemize}

\section{Security Configuration}
\label{sec:security}

Security in SIPMS is implemented as a layered model:
(1) application authentication and authorization, and (2) database least-privilege access.

\subsection{Application authentication}
SIPMS requires users to authenticate before accessing protected modules. The backend issues
a signed access token after validating credentials against the \texttt{app\_user} table (passwords
are stored as hashes). The frontend stores the session token on the client side and attaches it
to subsequent API requests.

\paragraph{Session lifecycle.}
Typical flow:
\begin{enumerate}
  \item User submits username/password on the login page.
  \item Backend validates credentials and returns an access token and basic user profile
  (user id, name, role).
  \item Frontend stores the token and user context, enabling authenticated navigation.
  \item Logout clears stored session data and returns the UI to the login screen.
\end{enumerate}

\subsection{Role-based authorization (RBAC) at the API layer}
After authentication, authorization is enforced using role-based permissions aligned with the
RBAC matrix (ADMIN / MANAGER / CLERK). At the API layer:

\begin{itemize}
  \item \textbf{IsAuthenticated:} required for most endpoints.
  \item \textbf{Privileged actions:} operations such as purchase order creation/receiving, cancellations,
  and transfers are restricted to ADMIN/MANAGER.
  \item \textbf{Clerk restrictions:} clerk users are typically limited to read-only access on master data
  and can create sales orders but cannot perform administrative or high-impact inventory actions.
\end{itemize}

This approach keeps business rules close to the workflow endpoints and prevents accidental exposure
of privileged operations.

\subsection{Database security and least-privilege access}
\label{subsec:db-security}

In addition to application RBAC, SIPMS configures database users with least privilege to provide
defense in depth. The script \texttt{security.sql} creates three privilege tiers:

\begin{itemize}
  \item \textbf{sipms\_admin:} full privileges on the \texttt{sipms} schema (schema management and full access).
  \item \textbf{sipms\_manager:} CRUD privileges on core business tables and permission to execute critical
  stored procedures.
  \item \textbf{sipms\_staff:} restricted privileges (read access to master data and limited inserts for sales
  order workflows; execute permission for sales confirmation procedure where applicable).
\end{itemize}

\subsection{SQL Listing: \texttt{security.sql}}
\lstinputlisting[style=sqlstyle, language=SQL,label={lst:security}]{../../database/security.sql}

\subsection{Operational notes and recommended hardening}
The repository uses placeholder passwords in SQL scripts; real credentials should be provided via
deployment configuration (environment variables/secrets). Additional practical hardening steps include:

\begin{itemize}
  \item rotate credentials regularly and restrict database access by host/network;
  \item avoid using admin DB users for the application runtime connection;
  \item enable audit logging for sensitive actions (either at application layer or database layer);
  \item ensure backups and recovery procedures are tested.
\end{itemize}

\newpage

%==============CHAPTER 6: Testing and Validation============

\chapter{Testing and Validation}
\label{chap:testing}

This chapter presents the testing approach and validation evidence for SIPMS.
Because Chapter~3 already demonstrates the implemented UI features, this chapter
focuses on correctness and database enforcement using Postman API tests and
MySQL Workbench verification queries (constraints, triggers, procedures, and views).

% ----------------------------------------------------------
\section{Test Environment and Tools}
\label{sec:test-environment}

We validate SIPMS using a black-box integration approach across three layers:

\begin{itemize}
  \item \textbf{Web UI (React):} sanity checks for key workflows and role-based restrictions.
  \item \textbf{Postman:} API-level testing for request/response validation, authentication, and RBAC.
  \item \textbf{MySQL Workbench:} database-level verification (tables, procedures, triggers, views) and integrity tests.
\end{itemize}

\paragraph{Dataset.}
A reproducible demo dataset is loaded using \texttt{seed\_data.sql}. This provides
sample users (ADMIN/MANAGER/CLERK), products, locations, and baseline records
to support repeatable testing.

% ----------------------------------------------------------
\section{Deployment and Test Setup Procedure}
\label{sec:test-setup}

\subsection{Database deployment}
The database is deployed by executing SQL scripts in the following order:

\begin{enumerate}
  \item \texttt{schema.sql} (create database \texttt{sipms}, tables, constraints)
  \item \texttt{seed\_data.sql} (demo/testing dataset)
  \item \texttt{views.sql} (analytics/reporting views)
  \item \texttt{procedures.sql} (stored procedures)
  \item \texttt{triggers.sql} (inventory enforcement triggers)
  \item \texttt{indexes.sql} (performance indexes)
  \item \texttt{security.sql} (optional, DB user grants; passwords replaced before execution)
\end{enumerate}

\subsection{API and UI setup}
The backend service is started and connected to the deployed MySQL schema.
The frontend is started and configured to send authenticated requests to the backend.

\subsection{Postman environment}
A Postman environment is configured with:
\texttt{base\_url}, and tokens for three test users (ADMIN/MANAGER/CLERK).
Login is performed via \texttt{/api/auth/login/} and the issued token is attached to
subsequent requests.

% ----------------------------------------------------------
\section{Test Strategy}
\label{sec:test-strategy}

Testing targets three categories:

\begin{itemize}
  \item \textbf{Authentication and RBAC:} verify login and role-based endpoint restrictions.
  \item \textbf{Workflow correctness:} validate purchase receiving and sales order processing (inventory sufficiency and automatic confirmation).
  \item \textbf{Database integrity enforcement:} validate triggers/procedures prevent invalid states
  (quantity $>0$, non-negative inventory) and reporting views return consistent results.
\end{itemize}

% ----------------------------------------------------------
\section{Test Cases and Results}
\label{sec:test-cases}

\subsection{Authentication and Role-Based Access Control}
\label{subsec:test-auth-rbac}

\begin{itemize}
  \item \textbf{Login (ADMIN/MANAGER/CLERK):} token is issued and stored on the client.
  \item \textbf{RBAC negative test:} CLERK is forbidden from privileged operations such as creating purchase orders.
  \item \textbf{RBAC positive test:} MANAGER/ADMIN can execute privileged operations successfully.
\end{itemize}

\begin{figure}[H]
  \centering
  \includegraphics[width=0.95\textwidth]{demo/test1a.png}
  \label{fig:test-login-users}
\end{figure}

\begin{figure}[H]
  \centering
  \includegraphics[width=0.95\textwidth]{demo/test1b.png}
  \label{fig:test-login-users}
\end{figure}

\begin{figure}[H]
  \centering
  \includegraphics[width=0.95\textwidth]{demo/test1c.png}
  \caption{Postman login tests: access token issued for ADMIN, MANAGER, and CLERK users.}
  \label{fig:test-login-users}
\end{figure}

\begin{figure}[H]
  \centering
  \includegraphics[width=0.95\textwidth]{demo/test2.png}
  \caption{RBAC negative test: CLERK user is denied when attempting to create a Purchase Order (HTTP 403).}
  \label{fig:test-rbac-clerk-po}
\end{figure}

\begin{figure}[H]
  \centering
  \includegraphics[width=0.95\textwidth]{demo/test3.png}
  \caption{RBAC positive test: ADMIN/MANAGER can create a Purchase Order successfully (HTTP 201).}
  \label{fig:test-rbac-admin-po}
\end{figure}

% ----------------------------------------------------------
\subsection{Purchase Receiving and Inventory Increase}
\label{subsec:test-po-receive}

This test validates that receiving goods records stock movements and automatically updates
\texttt{inventory\_level} via the \texttt{AFTER INSERT} trigger on \texttt{stock\_movement}.

\begin{itemize}
  \item Create a Purchase Order (MANAGER/ADMIN).
  \item Execute \texttt{/api/purchase-orders/<po\_id>/receive-all/}.
  \item Verify a \texttt{PURCHASE\_RECEIPT} movement is inserted and inventory increases at the destination location.
\end{itemize}

\begin{figure}[H]
  \centering
  \includegraphics[width=0.95\textwidth]{demo/test4a.png}
  \caption{Receive-all workflow: stock receipt succeeds and PO status is updated.}
  \label{fig:test-receive-all}
\end{figure}

\begin{figure}[H]
  \centering
  \includegraphics[width=0.95\textwidth]{demo/test4b.png}
  \caption{Workbench verification: inventory level increases after \texttt{PURCHASE\_RECEIPT} movement insert.}
  \label{fig:test-inventory-increase}
\end{figure}

% ----------------------------------------------------------
\subsection{Sales Order Validation and Automatic Confirmation}
\label{subsec:test-sales-confirm}

In SIPMS, sales orders are validated and confirmed at creation time. The backend calls the
stored procedure \texttt{sp\_confirm\_sales\_order} as part of the sales order creation workflow:
it checks inventory sufficiency before inserting \texttt{SALES\_ISSUE} movements and updating
\texttt{inventory\_level} through triggers. Therefore, insufficient stock is observed as a
\emph{create-time rejection} (HTTP 400) rather than a separate ``confirm later'' step.

\paragraph{PASS case (sufficient stock).}
Action: create a sales order via \texttt{POST /api/sales-orders/} with requested quantities within available stock.\\
Expected: the sales order is created in \texttt{CONFIRMED} status, \texttt{SALES\_ISSUE} movements are inserted,
and inventory decreases automatically.

\begin{figure}[H]
  \centering
  \includegraphics[width=0.95\textwidth]{demo/test5.png}
  \caption{Sales order PASS case: creation succeeds and the order is \texttt{CONFIRMED}; inventory is reduced accordingly.}
  \label{fig:test-sales-confirm-pass}
\end{figure}

\paragraph{FAIL case (insufficient stock).}
Action: attempt to create a sales order where at least one line item exceeds on-hand quantity.\\
Expected: the request is rejected (HTTP 400) with a database validation error (e.g., \texttt{ERROR 1644}),
no sales order is created, and inventory remains unchanged.

\begin{figure}[H]
  \centering
  \includegraphics[width=0.95\textwidth]{demo/test6.png}
  \caption{Sales order FAIL case: creation is rejected due to insufficient stock (HTTP 400 with DB error).}
  \label{fig:test-sales-confirm-fail}
\end{figure}

% ----------------------------------------------------------
\subsection{Direct Trigger Validation}
\label{subsec:test-trigger-invariants}

To demonstrate defense-in-depth, we validate database triggers directly in MySQL Workbench
by inserting into \texttt{stock\_movement} (bypassing the API layer).

Two critical invariants are enforced directly in the database:

\begin{itemize}
  \item \textbf{Quantity must be $> 0$:} invalid movements are blocked by a \texttt{BEFORE INSERT} trigger.
  \item \textbf{Non-negative inventory:} movements that would result in negative stock are rejected by
  the \texttt{AFTER INSERT} trigger.
\end{itemize}

\begin{figure}[H]
  \centering
  \includegraphics[width=0.95\textwidth]{demo/test7.png}
  \caption{Trigger validation: \texttt{stock\_movement.quantity} must be strictly positive.}
  \label{fig:test-trigger-positive-qty}
\end{figure}

\begin{figure}[H]
  \centering
  \includegraphics[width=0.95\textwidth]{demo/test8.png}
  \caption{Trigger validation: movements that would cause negative inventory are rejected.}
  \label{fig:test-trigger-nonnegative}
\end{figure}

% ----------------------------------------------------------
\subsection{Reporting Views Validation}
\label{subsec:test-views}

Reporting views are validated by executing read queries and comparing results with the
reporting pages in the web interface (without duplicating UI screenshots from Chapter~3).

\begin{figure}[H]
  \centering
  \includegraphics[width=0.95\textwidth]{demo/test9.png}
  \caption{Workbench validation: \texttt{view\_low\_stock\_products} returns products below reorder level.}
  \label{fig:test-view-low-stock}
\end{figure}

% ----------------------------------------------------------
\section{Summary}
\label{sec:test-summary}

Overall, SIPMS testing confirms:
(i) role-based access restrictions are enforced at the API layer,
(ii) business workflows correctly generate stock movements and update inventory,
and (iii) database triggers/procedures prevent invalid inventory states (negative stock, invalid quantities),
while views support consistent reporting.

\end{document}


% \begin{wrapfigure}{r}{0.3\textwidth}
%     \centering
%     \vspace{-20pt} % Adjust the space above the image
%     \includegraphics[width=0.28\textwidth]{bai 1.png}
%     \vspace{-10pt} % Adjust the space below the image
%     % \caption{Mô tả ảnh ở đây}
%     % \label{fig:your_label}
% \end{wrapfigure}


 
% \begin{figure}[h!]
% \centering
% \includegraphics[width=0.6\textwidth]{Problem5.png}
% \end{figure}